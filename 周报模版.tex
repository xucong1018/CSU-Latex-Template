\documentclass[UTF8]{article}
\usepackage{CTEX}
\usepackage{datetime}  
\usepackage{advdate}  
\usepackage{graphicx}  
\usepackage{geometry}
\usepackage{sectsty}
\sectionfont{\centering}
\geometry{a4paper,scale=0.8}
\usepackage{fancyhdr}
\pagestyle{fancy}
 
\begin{document}
	\lhead{}
	\chead{}
	\rhead{\bfseries 中南大学大数据组}
	\lfoot{}
	\cfoot{\thepage}
	\rfoot{}
	\renewcommand{\headrulewidth}{0.4pt}
	\renewcommand{\footrulewidth}{0pt} 
	\title{\LARGE \textbf{论文阅读报告}}
	\author{作者:蒋博的小迷弟}
	\date{ \number\year 年 \number\month 月 \number\day 日}
	\maketitle
	\thispagestyle{fancy}
\begin{center} 
\section{论文信息}
\end{center}
\subsection{按参考文献格式给出,例如“作者全名, 论文标题, 期刊或会议名称, (期卷号), 页码, 发表时间”:}

\section{作者信息}
\large \textbf{主要作者信息及个人主页链接:}
\begin{enumerate}
\item{主要作者一:作者名、单位、个人主页链接、已发表的主要相关论文} \\

\item{主要作者二:作者名、单位、个人主页链接、已发表的主要相关论文}\\
\end{enumerate}

\section{论文内容}
\subsection{要解决的问题(清晰描述该论文解决的问题,简单示例说明问题的输入和输出)}
 
\subsection{别人的方法/主流的解决方法 (要求理解主流方法的主要思想和流程)}

\subsection{本文的方法(清晰说明方法的输入、输出以及过程中数据格式或规模的变化,据此描述方法具体细节,图文并茂,尽可能详尽、易懂)}

\subsection{优点、不足(描述方法创新点,以及分析存在不足。)} 

\subsection{数据集以及代码(介绍数据下载链接、数据规模,数据内容。介绍代码下载链接、精读的论文要跑通代码,对照代码讲解。)} 
\begin{enumerate}  
	\item{数据集}
	\item{代码}
\end{enumerate} 
\subsection{我们能做什么?(思考本文提出方法对当前科研工作的启发,不局限于以下几点)} 

\subsubsection{研究的问题是否可以扩展(例如研究不同领域背景相似问题、研究他们没有考虑到的问题…)?}

\subsubsection{论文的方法如何改进(在某些步骤上增强、增加某些步骤、改进方法框架…)?}
\begin{enumerate}  
	\item
	\item
		\item
			\item	\item
			
\end{enumerate} 

%\begin{figure}[htbp]  
%\centering
%\includegraphics[width=\linewidth]{framework.pdf}  
%\begin{minipage}[t]{0.3\textwidth}  
%\centering  
%\includegraphics{left}  
%\caption{part one}  
%\end{minipage}  
%\begin{minipage}[t]{0.3\textwidth}  
%\centering  
%\includegraphics{right}  
%\caption{part two}  
%\end{minipage}  
%\end{figure}

\newpage
\end{document} 